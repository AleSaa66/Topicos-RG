\documentclass[letterpaper,11pt]{article}
\usepackage[spanish]{babel}
\usepackage[utf8]{inputenc}
\usepackage{graphicx}
\usepackage{amsfonts,amsmath,amssymb,float, amsthm,mathrsfs}  
\usepackage[right=4.5cm,left=2cm,top=3cm,bottom=3cm,headsep= 0.7cm,footskip=0.5cm]{geometry}
\usepackage{enumerate}
\usepackage{wrapfig} 
\usepackage[rflt]{floatflt} 
\usepackage{framed}
%\usepackage[most]{tcolorbox}
\usepackage[dvipsnames]{xcolor}
\colorlet{shadecolor}{green!20}
\setlength\FrameSep{0.5ex}
\usepackage{thmtools}
\usepackage{esint}
\usepackage{cancel}
\usepackage{listings} 
\usepackage{pstricks, caption}
\usepackage[colorlinks]{hyperref}
\usepackage{csquotes}
\usepackage{fullpage}
\usepackage{enumitem}
\usepackage{etoolbox}
\usepackage{tikz}
\usepackage{tikz-3dplot}
\tdplotsetmaincoords{80}{70}
\usetikzlibrary{decorations.markings}
\usetikzlibrary{arrows,babel}
\usepackage[font=small]{caption}
\usepackage{scalerel} %\scaleto{text}{size}
\usepackage{subfigure}
\usepackage{fancyhdr}
\usepackage{comment}
\usepackage{marginnote}
\usepackage{tensor}
\usepackage{cleveref}
\newcommand{\dbar}{\mathchar'26\mkern-12mu d}
\renewcommand*{\marginnotevadjust}{-0.1cm}
\renewcommand*{\marginfont}{\footnotesize}
\setlength{\headheight}{15pt}
\addtolength{\topmargin}{-14.49998pt}
\setlength{\headsep}{15pt}
\setlength{\footskip}{14.49998pt}
\decimalpoint
\newcommand{\grad}{^\circ}
\newlength{\drop}
\DeclareMathOperator{\sign}{sgn}
\DeclareMathOperator{\Log}{Log}
\providecommand{\norm}[1]{\lVert#1\rVert}

\begin{document}

\pagestyle{plain}

\begin{flushleft}\vspace{-2cm}
Departamento de Física \\
Facultad de Cs. Físicas y Matemáticas\\
Universidad de Concepción
\end{flushleft}

\begin{flushright}\vspace{-1.5cm}
\textbf{Tópicos en Relatividad General} 
\end{flushright}

\rule{\linewidth}{0.1mm}

\begin{center}
\textbf{\LARGE Semana 1}
\end{center}

\begin{flushleft}
\textbf{Nombre:} Alejandro Saavedra San Martín. \\
\textbf{Profesor:} Guillermo Rubilar Alegría.
\end{flushleft}

La métrica de Schwarzschild exterior, en las coordenadas de curvatura $x^{\mu} = (ct,r,\theta,\varphi)$, toma la forma
\colorlet{shadecolor}{red!20}
\begin{shaded}
\begin{equation} \label{eq:Scwarzschild-metric}
    ds^2 = \left( 1 - \frac{2m}{r}\right) c^2 dt^2 - \frac{dr^2}{\left( 1- \frac{2m}{r} \right)} - r^2 \left[ d\theta^2 + \sin^2 \theta d\varphi^2 \right], \quad r > 2m.
\end{equation}
    
\end{shaded}

Buscamos escribir el $ds^2$ en un sistema coordenado isótropo en el que la sección espacial ($t = \text{cte}$) sea proporcional a la usual distancia euclideana:
\begin{equation}
    ds^2 = \Tilde{A} (c^2 dt^2) - \Tilde{B} d\ell_{R^{3}}^2,
\end{equation}
donde $d\ell_{R^{3}}^2 = dx^2 + dy^2 + dz^2$ en coordenadas cuasi-cartesianas, o $d\ell_{R^{3}}^2 = d\rho^2 + \rho^2 (d\theta^2 + \sin^2\theta d\varphi^2)$ en coordenadas cuasi-esféricas. Entonces, las coordenadas isotrópicas, $x^{\mu} = (ct, \rho, \theta, \varphi)$, que buscamos deben ser tales que el elemento de línea adopte la forma
\begin{equation} \label{eq:pre-Sch-metric-isotropic}
    ds^2 = \Tilde{A}(\rho) c^2 dt^2 - \Tilde{B}(\rho) \left[ d\rho^2 + \rho^2 (d\theta^2 + \sin^2\theta d\varphi^2) \right].
\end{equation}

Como $ds^2$ es un escalar bajo transformaciones generales de coordenadas (TGC's), comparando \eqref{eq:pre-Sch-metric-isotropic} con \eqref{eq:Scwarzschild-metric}, encontramos  las siguientes condiciones:
\begin{align}
    \Tilde{A}(\rho) &= 1 - \frac{2m}{r}, \label{eq:isotropic-1}\\
    \Tilde{B}(\rho) d\rho^2 &= \left( 1 - \frac{2m}{r} \right)^{-1} dr^2, \label{eq:isotropic-2} \\
    \Tilde{B}(\rho) \rho^2 &= r^2. \label{eq:isotropic-3}
\end{align}

Note que la condición \eqref{eq:isotropic-3} que debe cumplirse para $d\theta^2$ es la misma para $d\varphi^2$.

Resolvamos el sistema de ecuaciones \eqref{eq:isotropic-1}-\eqref{eq:isotropic-3} considerando como incógnitas las funciones $\Tilde{A}$, $\Tilde{B}$ y $\rho = \rho(r)$. De la ecuación \eqref{eq:isotropic-3},
\begin{equation} \label{eq:isotropic-4}
    \Tilde{B}(\rho) =  \frac{r^2}{\rho^2}.
\end{equation}

Reemplazando en \eqref{eq:isotropic-2}, encontramos una ecuación diferencial para $\rho(r)$ que resulta ser separable:
\begin{equation} \label{eq:isotropic-5}
    \frac{r^2}{\rho^2} d\rho^2 = \left( 1 - \frac{2m}{r}\right)^{-1} dr^2.
\end{equation}

Si suponemos que $d\rho/dr > 0$ y $r > 2m$, podemos reescribir \eqref{eq:isotropic-5} como
\begin{equation} \label{eq:isotropic-6}
    \frac{d\rho}{\rho} = \frac{dr}{\sqrt{r(r-2m)}}.
\end{equation}

Integrando a ambos lados de la ecuación \eqref{eq:isotropic-6}:
\begin{equation}
    \int \frac{d\rho}{\rho} = \ln(\rho) = \int \frac{dr}{\sqrt{r(r-2m)}}. \label{eq:isotropic-7}
\end{equation}

Para resolver la integral en $r$, usemos la sustitución trigonométrica 
\begin{equation}
    r - m = m \sec\theta \Rightarrow dr = m \sec\theta \tan\theta d\theta,
\end{equation}
válida para $0 < \theta < \pi/2$ \footnote{De esta forma se recupera el dominio $r > 2m$.}. Entonces, usando la identidad trigonométrica $\tan^2\theta + 1 \equiv \sec^2\theta$, encontramos que
\begin{align}
    \int \frac{dr}{\sqrt{r(r-2m)}} &= \int \frac{m \sec\theta \tan\theta d\theta}{\sqrt{m^2 \sec^2\theta - m^2}} \nonumber \\
    &= \int \frac{\sec\theta \tan \theta}{\sqrt{\sec^2\theta - 1}} d\theta \nonumber \\
    &= \int \sec\theta \,d\theta \nonumber \\
    &= \ln|\sec\theta + \tan \theta| + C \nonumber \\
    &= \ln\left|\frac{r-m}{m} + \frac{\sqrt{r^2 - 2mr}}{m} \right| + C \nonumber \\
    &= \ln\left(r -m +\sqrt{r^2-2mr}\right) + D, \label{eq:isotropic-8}
\end{align}
donde $D = C- \ln(m)$ es una constante de integración. Por lo tanto, comparando las ecuaciones \eqref{eq:isotropic-7} y \eqref{eq:isotropic-8}, obtenemos 
\begin{equation}
    \ln(\rho) = \ln\left(r -m +\sqrt{r^2-2mr}\right) + D \Rightarrow \rho = e^{D} \left(r -m +\sqrt{r^2-2mr}\right).
\end{equation}

Despejando la coordenada $r$ en términos de $\rho$:
\begin{align}
    (\rho + e^D m) - e^D r &= e^D \sqrt{r^2-2mr} \\
    [(\rho + e^D m) - e^D r]^2 &= e^{2D} (\sqrt{r^2-2mr} )^2\\
    (\rho + e^D m)^2 - 2 e^D r (\rho + e^D m) + \cancel{e^{2D} r^2} &= e^{2D} (\cancel{r^2} - 2mr) \\
    (\rho + e^D m)^2 &= 2 e^D r(\rho + \cancel{e^{D} m}) - \cancel{2mr e^{2D}} \\
    (\rho + e^D m)^2 &= 2 e^D r \rho \\
    \frac{(\rho + e^D m)^2}{2 e^D \rho} &= r \\
    \left(\frac{1}{2e^D}\right) \rho \left[ 1 + \frac{m}{2\rho} \left( 2 e^D \right) \right]^2 &= r.
\end{align}

Si denotamos la constante $\beta = 1/(2e^D)$, hemos encontrado la transformación de coordenadas entre $\rho$ y $r$:
\begin{equation} 
    r = \beta \rho \left( 1+ \frac{m}{2\beta \rho} \right)^2.
\end{equation}

La constante $\beta$ está relacionada con la escala elegida para la coordenadas radial $\rho$. Sin perder generalidad, podemos elegir esta constante de modo que asintóticamente se aproxime a la usual coordenada radial esférica en un espacio plano. Esto requiere que $\beta = 1$. Así, la transformación de coordenadas queda 
\begin{equation}
r = \rho \left( 1 + \frac{m}{2\rho} \right)^2.
\end{equation}

Considerando $\rho = \rho(r)$ y derivando implícitamente con respecto a $r$, tenemos que
\begin{align}
1 &= \frac{d}{dr} \left[ \rho \left( 1 + \frac{m}{2\rho} \right)^2 \right] \nonumber \\
 &= \frac{d\rho}{dr} \left( 1 + \frac{m}{2\rho} \right)^2 + 2 \rho \left( 1 + \frac{m}{2\rho} \right) \left( - \frac{m}{2\rho^2} \right) \frac{d\rho}{dr} \nonumber\\
 &= \frac{d\rho}{dr} \left(1 + \frac{m}{2\rho} \right) \left[ \left(1 + \frac{m}{2\rho} \right) - \frac{m}{\rho}\right] \nonumber\\
 &= \frac{d\rho}{dr} \left(1 + \frac{m}{2\rho} \right)  \left(1 - \frac{m}{2\rho} \right)\nonumber\\
 &= \frac{d\rho}{dr} \left(1 - \frac{m^2}{4\rho^2} \right). 
\end{align}

Luego, 
\begin{equation}
\frac{d\rho}{dr} = \frac{1}{ \left(1 - \frac{m^2}{4\rho^2} \right)}.
\end{equation}

Recordemos que para escribir la ecuación \eqref{eq:isotropic-6} hemos supuesto que  $d\rho/dr > 0$. Entonces, debe cumplirse que 
\begin{equation}
1 - \frac{m^2}{4\rho^2} > 0 \Leftrightarrow \rho >  \frac{m}{2}.
\end{equation}

Por lo tanto, la transformación entre las coordenadas $\rho$ y $r$, con $r > 2m$, es
\colorlet{shadecolor}{green!20}
\begin{shaded}
\begin{equation} \label{eq:isotropic-r(rho)}
    r = \rho \left( 1 + \frac{m}{2\rho} \right)^2, \quad \rho > \frac{m}{2}.
\end{equation}    
\end{shaded}

Si reemplazamos \eqref{eq:isotropic-r(rho)} en \eqref{eq:isotropic-4}, obtenemos el coeficiente métrico 
\begin{shaded}
\begin{equation} \label{eq:metric-isotropic-B}
    \Tilde{B}(\rho) = \left( 1 + \frac{m}{2\rho} \right)^4.
\end{equation}    
\end{shaded}

Por último, al reemplazar \eqref{eq:isotropic-r(rho)} en \eqref{eq:isotropic-1}, obtenemos el coeficiente métrico restante:
\begin{align}
    \Tilde{A}(\rho) &= 1 - \frac{2m/\rho}{\left( 1 + \frac{m}{2\rho}\right)^2} \nonumber \\
    &= \frac{\left( 1 + \frac{m}{2\rho}\right)^2 - \frac{2m}{\rho}}{\left( 1 + \frac{m}{2\rho}\right)^2}\nonumber\\
    &= \frac{\left(1 - \frac{m}{2\rho} \right)^2}{\left( 1 + \frac{m}{2\rho}\right)^2},
\end{align}
es decir,
\begin{shaded}
    \begin{equation}
        \Tilde{A}(\rho) = \frac{\left(1 - \frac{m}{2\rho} \right)^2}{\left( 1 + \frac{m}{2\rho}\right)^2}. 
    \end{equation}
\end{shaded}

Resumiendo, el elemento de línea de la solución de Schwarzschild exterior en coordenadas isotrópicas tiene la forma:
\colorlet{shadecolor}{red!20}
\begin{shaded}
    \begin{equation} \label{eq:metric-Schwarz-isotropic-sphere}
        ds^2 = \frac{\left(1 - \frac{m}{2\rho} \right)^2}{\left( 1 + \frac{m}{2\rho}\right)^2} c^2 dt^2 - \left( 1 + \frac{m}{2\rho} \right)^4 \left[d\rho^2 + \rho^2 (d\theta^2 + \sin^2\theta d\varphi^2)\right], \quad \rho > \frac{m}{2}.
    \end{equation}
\end{shaded}

En coordenadas cuasi-cartesianas, bajo la transformación de coordenadas
\begin{align}
    x &= \rho \sin\theta \cos\varphi, \label{eq:esferica-carte-1} \\
    y &= \rho \sin\theta \sin\varphi, \label{eq:esferica-carte-2}\\
    z &= \rho \cos\theta, \label{eq:esferica-carte-3}
\end{align}
el elemento de línea también puede ser escrito como
\begin{shaded}
    \begin{equation}\label{eq:metric-Schwarz-isotropic-carte}
        ds^2 = \frac{\left(1 - \frac{m}{2\rho} \right)^2}{\left( 1 + \frac{m}{2\rho}\right)^2} c^2 dt^2 - \left( 1 + \frac{m}{2\rho} \right)^4 \left[dx^2 + dy^2 + dz^2\right], \quad \rho = \sqrt{x^2 + y^2 + z^2}.
    \end{equation}
\end{shaded}

En efecto, calculando los diferenciales de la transformación de coordenadas \eqref{eq:esferica-carte-1}-\eqref{eq:esferica-carte-3}:
\begin{align}
    dx^2 &= (\sin\theta \cos\varphi \,d\rho^2 + \rho \cos\theta\cos\varphi \,d\theta - \rho \sin\theta \sin \varphi \,d\varphi)^2 \nonumber\\
    &= {\color{red} \sin^2\theta \cos^2\varphi \,d\rho^2}  {\color{blue} + \rho^2 \cos^2\theta \cos^2\varphi \,d\theta^2} {\color{ForestGreen} + \rho^2 \sin^2\theta \sin^2\varphi \,d\varphi^2} + 2\rho \sin\theta \cos\theta \cos^2\varphi \,d\rho\theta \nonumber\\
    & \quad  - 2\rho \sin^2\theta \sin\varphi \cos\varphi \,d\rho d\varphi - 2\rho^2 \sin\theta \cos\theta \sin\varphi \cos\varphi \,d\theta d\varphi,  \label{eq:esferica-carte-4}\\
    dy^2 &= (\sin\theta \sin\varphi \,d\rho^2 + \rho \cos\theta\sin\varphi \,d\theta + \rho \sin\theta \cos \varphi \,d\varphi)^2 \nonumber\\
    &= {\color{red}\sin^2\theta \sin^2\varphi \,d\rho^2}  {\color{blue} + \rho^2 \cos^2\theta \sin^2\varphi \,d\theta^2}  {\color{ForestGreen} + \rho^2 \sin^2\theta \cos^2\varphi\,d\varphi^2} + 2 \rho\sin\theta \cos\theta \sin^2\varphi \,d\rho d\theta \nonumber \\
    & \quad  + 2 \rho\sin^2\theta \sin\varphi \cos\varphi \,d\rho 
    d\varphi + 2 \rho^2 \sin\theta \cos\theta \sin\varphi \cos \varphi \,d\theta d\varphi, \label{eq:esferica-carte-5}\\
    dz^2 &= (\cos\theta \,d\rho - \rho\sin \theta \,d\theta)^2 \nonumber\\
    &= {\color{red}\cos^2\theta \,d\rho^2} - 2 \rho\sin\theta \cos\theta \,d\rho d\theta  {\color{blue} +
    \rho^2 \sin^2\theta \,d\theta^2}. \label{eq:esferica-carte-6}
\end{align}

 Entonces, al calcular el elemento de línea $d\ell^2_{R^3} = dx^2 + dy^2 + dz^2$, en coordenadas esféricas, los términos en colores de las ecuaciones \eqref{eq:esferica-carte-5}-\eqref{eq:esferica-carte-6} se pueden agrupar, mientras que los términos restantes de cancelan mutuamente. Por lo tanto, 
\begin{equation}
    d\ell^2_{R^3} = dx^2 + dy^2 + dz^2 = d \rho^2 + \rho^2 d\theta^2 + \rho^2 \sin^2\theta d\varphi^2.
\end{equation}

Al igual que la métrica \eqref{eq:Scwarzschild-metric}, las métricas \eqref{eq:metric-Schwarz-isotropic-sphere} y \eqref{eq:metric-Schwarz-isotropic-carte} también satisfacen las ecuaciones de Einstein en el vacío. Para mayores detalles puede consultar en el siguiente \href{https://github.com/AleSaa66/Topicos-RG/blob/main/Semana%201/Semana-1.ipynb}{Notebook}.



\end{document}

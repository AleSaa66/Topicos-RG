\documentclass[letterpaper,11pt]{article}
\usepackage[spanish]{babel}
\usepackage[utf8]{inputenc}
\usepackage{graphicx}
\usepackage{amsfonts,amsmath,amssymb,float, amsthm,mathrsfs}  
\usepackage[right=4.5cm,left=2cm,top=3cm,bottom=3cm,headsep= 0.7cm,footskip=0.5cm]{geometry}
\usepackage{enumerate}
\usepackage{wrapfig} 
\usepackage[rflt]{floatflt} 
\usepackage{framed}
%\usepackage[most]{tcolorbox}
\usepackage[dvipsnames]{xcolor}
\colorlet{shadecolor}{green!20}
\setlength\FrameSep{0.5ex}
\usepackage{thmtools}
\usepackage{esint}
\usepackage{cancel}
\usepackage{listings} 
\usepackage{pstricks, caption}
\usepackage[colorlinks]{hyperref}
\usepackage{csquotes}
\usepackage{fullpage}
\usepackage{enumitem}
\usepackage{etoolbox}
\usepackage{tikz}
\usepackage{tikz-3dplot}
\tdplotsetmaincoords{80}{70}
\usetikzlibrary{decorations.markings}
\usetikzlibrary{arrows,babel}
\usepackage[font=small]{caption}
\usepackage{scalerel} %\scaleto{text}{size}
\usepackage{subfigure}
\usepackage{fancyhdr}
\usepackage{comment}
\usepackage{marginnote}
\usepackage{tensor}
\usepackage{cleveref}
\newcommand{\dbar}{\mathchar'26\mkern-12mu d}
\renewcommand*{\marginnotevadjust}{-0.1cm}
\renewcommand*{\marginfont}{\footnotesize}
\setlength{\headheight}{15pt}
\addtolength{\topmargin}{-14.49998pt}
\setlength{\headsep}{15pt}
\setlength{\footskip}{14.49998pt}
\decimalpoint
\newcommand{\grad}{^\circ}
\newlength{\drop}
\DeclareMathOperator{\sign}{sgn}
\DeclareMathOperator{\Log}{Log}
\providecommand{\norm}[1]{\lVert#1\rVert}

\let\cancelorigcolor\CancelColor% Just for conveniency...

\newcommand{\CancelTo}[3][]{%
  \ifblank{#1}{}{%
    \renewcommand{\CancelColor}{#1}%
  }
  \cancelto{#2}{#3}% 
}


\begin{document}

\pagestyle{plain}

\begin{flushleft}\vspace{-2cm}
Departamento de Física \\
Facultad de Cs. Físicas y Matemáticas\\
Universidad de Concepción
\end{flushleft}

\begin{flushright}\vspace{-1.5cm}
\textbf{Tópicos en Relatividad General} 
\end{flushright}

\rule{\linewidth}{0.1mm}

\begin{center}
\textbf{\LARGE Semana 2}
\end{center}

\begin{flushleft}
\textbf{Nombre:} Alejandro Saavedra San Martín. \\
\textbf{Profesor:} Guillermo Rubilar Alegría.
\end{flushleft}

La métrica de Schwarzschild exterior, en las coordenadas de curvatura $x^{\mu} = (ct,r,\theta,\varphi)$, toma la forma
\colorlet{shadecolor}{red!20}
\begin{shaded}
\begin{equation} \label{eq:Scwarzschild-metric}
    ds^2 = \left( 1 - \frac{2m}{r}\right) c^2 dt^2 - \frac{dr^2}{\left( 1- \frac{2m}{r} \right)} - r^2 \left[ d\theta^2 + \sin^2 \theta d\varphi^2 \right], \quad r > 2m.
\end{equation}
    
\end{shaded}

Las componentes no nulas de los símbolos de Christoffel son:
\begin{align}
    \Gamma\indices{^0_{01}} &= \frac{m}{r(r-2m)}, \quad \Gamma\indices{^1_{00}} = \frac{(r-2m)m}{r^3}, \quad \Gamma\indices{^1_{11}} = - \frac{m}{r(r-2m)}, \\
    \Gamma\indices{^1_{22}} &= - (r-2m), \quad \Gamma\indices{^1_{33}} = - (r-2m) \sin^2\theta, \quad \Gamma\indices{^2_{12}} = \frac{1}{r}, \\
    \Gamma\indices{^2_{33}} &= - \sin\theta \cos\theta, \quad \Gamma\indices{^3_{13}} = \frac{1}{r}, \quad \Gamma\indices{^3_{23}} = \frac{\cos\theta}{\sin\theta}.   
\end{align}

Cálculo de los símbolos de Christoffel en el siguiente \href{https://github.com/AleSaa66/Topicos-RG/blob/main/Semana%202/Semana-2.ipynb}{Notebook}.

\section{Geodésicas tipo tiempo} 

Si parametrizamos las geodésicas usando el tiempo propio $\tau$ (parámetro afín), $x^{\mu} = x^{\mu}(\tau)$, la ecuación de la geodésica toma la forma:
\begin{equation}
    \frac{d^2 x^{\mu}}{d\tau^2} + \Gamma\indices{^\mu_{\nu \lambda}} \frac{dx^{\nu}}{d\tau} \frac{dx^{\lambda}}{d\tau} = 0.
\end{equation}

Para la componente $\mu = 0$, la ecuación nos queda
\begin{align}
\frac{d^2 x^{0}}{d\tau^2} + \Gamma\indices{^0_{\nu \lambda}} \frac{dx^{\nu}}{d\tau} \frac{dx^{\lambda}}{d\tau} &= 0 \\
\frac{d^2 x^{0}}{d\tau^2} +  \Gamma\indices{^0_{01}} \frac{dx^{0}}{d\tau} \frac{dx^1}{d\tau} +  \Gamma\indices{^0_{10}} \frac{dx^{1}}{d\tau} \frac{dx^0}{d\tau} &= 0 \\
\frac{d^2 x^{0}}{d\tau^2} +  2 \Gamma\indices{^0_{01}} \frac{dx^{0}}{d\tau} \frac{dx^1}{d\tau}  &= 0  \\
c \Ddot{t} +  \frac{2m}{r(r-2m)} c\dot{t} \dot{r}&= 0 \\
\Ddot{t} +  \frac{2m}{r(r-2m)} \dot{t} \dot{r}&= 0,
\end{align}
donde hemos denotado $\dot{f} := df/d\tau$.

Para la componente $\mu = 1$, la ecuación nos queda
\begin{align}
\frac{d^2 x^{1}}{d\tau^2} + \Gamma\indices{^1_{\nu \lambda}} \frac{dx^{\nu}}{d\tau} \frac{dx^{\lambda}}{d\tau} &= 0 \\
    \frac{d^2 x^{1}}{d\tau^2} +  \Gamma\indices{^1_{00}} \left( \frac{dx^{0}}{d\tau} \right)^2 +  \Gamma\indices{^1_{11}} \left( \frac{dx^{1}}{d\tau} \right)^2 + \Gamma\indices{^1_{22}} \left( \frac{dx^2}{d\tau}\right)^2 +  \Gamma\indices{^1_{33}} \left( \frac{dx^{3}}{d\tau} \right)^2 &= 0 \\
\Ddot{r} + \frac{(r-2m)m}{r^3} c^2 \dot{t}^2 - \frac{m}{r(r-2m)} \dot{r}^2 - (r-2m) \dot{\theta}^2 - (r-2m) \sin^2\theta \dot{\varphi}^2 &= 0 \\
\Ddot{r} + \frac{mc^2(r-2m)}{r^3} \dot{t}^2 - \frac{m}{r(r-2m)} \dot{r}^2 - (r-2m) \left[\dot{\theta}^2 + \sin^2\theta \dot{\varphi}^2 \right]  &= 0.
\end{align}

Para la componente $\mu = 2$, la ecuación nos queda\begin{align}
\frac{d^2 x^{2}}{d\tau^2} + \Gamma\indices{^2_{\nu \lambda}} \frac{dx^{\nu}}{d\tau} \frac{dx^{\lambda}}{d\tau} &= 0 \\
    \frac{d^2 x^{2}}{d\tau^2} +  \Gamma\indices{^2_{12}}  \frac{dx^{1}}{d\tau} \frac{dx^{2}}{d\tau} +  \Gamma\indices{^2_{21}}  \frac{dx^{2}}{d\tau} \frac{dx^{1}}{d\tau} + \Gamma\indices{^2_{33}} \left( \frac{dx^{3}}{d\tau} \right)^2  &= 0 \\
    \frac{d^2 x^{2}}{d\tau^2} +  2 \Gamma\indices{^2_{12}}  \frac{dx^{1}}{d\tau} \frac{dx^{2}}{d\tau} + \Gamma\indices{^2_{33}} \left( \frac{dx^{3}}{d\tau} \right)^2  &= 0 \\
\Ddot{\theta} + \frac{2}{r} \dot{r} \dot{\theta} - \sin\theta\cos\theta \dot{\varphi}^2 &= 0.
\end{align}

Para la componente $\mu = 3$, la ecuación nos queda\begin{align}
\frac{d^2 x^{3}}{d\tau^2} + \Gamma\indices{^3_{\nu \lambda}} \frac{dx^{\nu}}{d\tau} \frac{dx^{\lambda}}{d\tau} &= 0 \\
    \frac{d^2 x^{3}}{d\tau^2} +  \Gamma\indices{^3_{13}}  \frac{dx^{1}}{d\tau} \frac{dx^{3}}{d\tau} +   \Gamma\indices{^3_{31}}  \frac{dx^{3}}{d\tau} \frac{dx^{1}}{d\tau} + \Gamma\indices{^3_{23}}  \frac{dx^{2}}{d\tau} \frac{dx^{3}}{d\tau} + \Gamma\indices{^3_{32}}  \frac{dx^{3}}{d\tau} \frac{dx^{2}}{d\tau}  &= 0 \\
    \frac{d^2 x^{3}}{d\tau^2} +  2 \Gamma\indices{^3_{13}}  \frac{dx^{1}}{d\tau} \frac{dx^{3}}{d\tau}  + 2\Gamma\indices{^3_{23}}  \frac{dx^{2}}{d\tau} \frac{dx^{3}}{d\tau}   &= 0 \\
    \Ddot{\varphi} + \frac{2}{r} \dot{r}\dot{\varphi} + 2 \frac{\cos\theta}{\sin \theta} \dot{\theta} \dot{\varphi} &= 0.
\end{align}

Por lo tanto, las ecuaciones de las geodésicas adoptan la siguiente forma explícita:
\colorlet{shadecolor}{green!20}
\begin{shaded}
\begin{align}
\Ddot{t} +  \frac{2m}{r(r-2m)} \dot{t} \dot{r}&= 0, \label{eq:geod-time-1} \\    \Ddot{r} + \frac{mc^2(r-2m)}{r^3} \dot{t}^2 - \frac{m}{r(r-2m)} \dot{r}^2 - (r-2m) \left[\dot{\theta}^2 + \sin^2\theta \dot{\varphi}^2 \right]  &= 0, \label{eq:geod-time-2} \\
\Ddot{\theta} + \frac{2}{r} \dot{r} \dot{\theta} - \sin\theta\cos\theta \dot{\varphi}^2 &= 0, \label{eq:geod-time-3} \\
 \Ddot{\varphi} + \frac{2}{r} \dot{r}\dot{\varphi} + 2 \frac{\cos\theta}{\sin \theta} \dot{\theta} \dot{\varphi} &= 0. \label{eq:geod-time-4}
\end{align}
\end{shaded}

Notemos que
\begin{align}
    \frac{1}{\left( 1 - \frac{2m}{r} \right)} \frac{d}{d\tau} \left[ \left( 1 -  \frac{2m}{r} \right) \dot{t} \right] &= \frac{1}{\left( 1 - \frac{2m}{r} \right)} \left[ \left( 1 - \frac{2m}{r} \right) \ddot{t} + \frac{2m}{r^2} \dot{t} \dot{r} \right] \nonumber \\
    &= \ddot{t} + \frac{2m}{r\left( r - 2m \right)} \dot{t} \dot{r}, \\
    \frac{1}{r^2 \sin^2\theta} \frac{d}{d\tau} \left[ r^2 \sin^2 \theta \dot{\varphi} \right] &= \frac{1}{r^2\sin^2\theta} \left[ r^2\sin^2\theta \ddot{\varphi} + 2r \sin^2\theta \dot{r} \dot{\varphi} + 2 r^2 \sin \theta \cos\theta \dot{\theta} \dot{\varphi}\right]\nonumber \\
    &= \ddot{\varphi} + \frac{2}{r} \dot{r} \dot{\varphi} + 2 \frac{\cos\theta}{\sin\theta} \dot{\theta} \dot{\varphi}.
\end{align}

Luego, las ecuaciones \eqref{eq:geod-time-1} y \eqref{eq:geod-time-4} pueden reescribirse como
\begin{align}
    \frac{1}{\left( 1 - \frac{2m}{r} \right)} \frac{d}{d\tau} \left[ \left( 1 -  \frac{2m}{r} \right) \dot{t} \right] &= 0, \\
     \frac{1}{r^2 \sin^2\theta} \frac{d}{d\tau} \left[ r^2 \sin^2 \theta \dot{\varphi} \right] &= 0.
\end{align}

En esta forma, es claro que ambas ecuaciones implican la existencia de dos constantes del movimiento, las cuales pueden obtenerse a partir de los vectores de Killing: $\xi_t = (1,0,0,0)$ y $\xi_{\varphi} = (0,0,0,1)$,  escrito en coordenadas $x^{\mu} = (ct,r,\theta,\varphi)$. En efecto, si $x^{\mu}(\tau)$ es la forma paramétrica de una geodésica, donde el parámetro afín escogido es el tiempo propio $\tau$ y $\xi^{\mu}$ un vector de Killing, la cantidad
\begin{equation}
    Q = g_{\mu\nu} \xi^{\mu} \dot{x}^{\nu}
\end{equation}

es constante a lo largo de la curva geodésica.

Para el Killing temporal:
\begin{equation}
    Q = g_{00} \xi_t^0 \dot{x}^0 = \left( 1 - \frac{2m}{r} \right) \dot{t}.
\end{equation}

Para el Killing en la variable $\varphi$:
\begin{equation}
    Q = g_{33} \xi_{\varphi}^3 \dot{x}^3 =  - r^2 \sin^2\theta  \dot{\varphi}.
\end{equation}

Por simplicidad introduciremos dos constantes adimensionales, $k$ y $h$, definidas por 
\colorlet{shadecolor}{green!20}
\begin{shaded}
\begin{equation} \label{eq:Cantidades-Conservadas}
k:= \left( 1 - \frac{2m}{r}\right) \dot{t}, \quad hmc := r^2 \sin^2\theta \dot{\varphi}.
\end{equation}
\end{shaded}

Estas cantidades coinciden con las definiciones newtonianas de energía y momentum angular por unidad de masa del cuerpo orbitando. Es claro de ver que la expresión para $\mathcal{L} := hmc$ coincide con la componente $z$ del momento angular, para el caso de la energía es necesario elaborar un poco más. Escribamos el tiempo propio como
\begin{equation}
c^2d\tau^2 = g_{\mu\nu} dx^{\mu}dx^{\nu} = \left(1 - \frac{2m}{r} \right) c^2 dt^2 - \frac{dr^2}{\left( 1 - \frac{2m}{r}\right)} - r^2 d\Omega^2 
\end{equation}
y al tomar el límite de campo débil $r \gg 2m$, 
\begin{equation}
\frac{1}{1 - \frac{2m}{r}} \approx 1.
\end{equation}

Así,
\begin{equation}
 c^2d\tau^2 \approx - \frac{2m}{r} c^2 dt^2 + c^2 dt^2 - d\vec{x} \cdot d\vec{x} = - \frac{2m}{r} c^2 dt^2 + \eta_{\mu\nu} dx^{\mu}dx^{\nu}.
\end{equation}

De la teoría de Relatividad Especial sabemos que 
\begin{equation}
\eta_{\mu\nu} dx^{\mu}dx^{\nu} = \left( 1 - \frac{v^2}{c^2}\right) c^2 dt^2.
\end{equation}

Entonces, 
\begin{equation}
c^2d\tau^2 \approx \left(1 - \frac{v^2}{c^2} - \frac{2m}{r} \right) c^2 dt^2 \Rightarrow \left(\frac{dt}{d\tau} \right)^2 \approx \frac{1}{1 - \frac{v^2}{c^2} - \frac{2m}{r}}.
\end{equation}

Evaluando la raíz cuadrada, encontramos que, para $r\gg 2m$,
\begin{align}
\dot{t} &\approx \frac{1}{\sqrt{1 - \frac{v^2}{c^2} - \frac{2m}{r}}} \nonumber \\
&= \frac{1}{\sqrt{1 -\frac{v^2}{c^2}}}  + \frac{1}{\left(\sqrt{1 -\frac{v^2}{c^2}  }\right)^3} \frac{m}{r} + \mathcal{O}\left(\frac{m^2}{r^2} \right) \nonumber\\
&= \gamma + \frac{m}{\gamma^3 r} + \mathcal{O}\left(\frac{m^2}{r^2} \right),
\end{align}
donde $\gamma = 1/\sqrt{1 - v^2/c^2}$. 

Si ahora consideramos el límite no relativista, $v \ll c$, 
\begin{equation}
\gamma \approx 1 + \frac{v^2}{2c^2}.
\end{equation}

Por lo tanto, en los límite de campo débil y no relativista,
\begin{align}
k &= \left(1 - \frac{2m}{r} \right) \dot{t} \nonumber\\
&\approx  \left(1 - \frac{2m}{r} \right) \left(\gamma + \frac{m}{\gamma^3 r} \right) \nonumber\\
&=  \gamma + \frac{m}{\gamma^3r} - \frac{2m}{r} \gamma - \cancelto{0}{\frac{2m^2}{\gamma^3 r^2}} \nonumber\\
&\approx 1 + \frac{v^2}{2c^2} + \frac{m}{(1 + \frac{v^2}{2c^2})^{3/2} r} - \frac{2m}{r} - \cancelto{0}{\frac{2m}{r} \frac{v^2}{2c^2}} \nonumber\\
&= 1 + \frac{v^2}{2c^2} + \frac{m}{r} \left( \frac{1}{(1 + \frac{v^2}{2c^2})^{3/2}} \right) - \frac{2m}{r} - \cancelto{0}{\frac{2m}{r} \frac{v^2}{2c^2}} \nonumber \\
&\approx 1 + \frac{v^2}{2c^2} + \frac{m}{r} \left( 1 - 3 \frac{v^2}{4c^2} \right) - \frac{2m}{r} - \cancelto{0}{\frac{2m}{r} \frac{v^2}{2c^2}}\nonumber \\
&\approx 1 + \frac{v^2}{2c^2} + \frac{m}{r} - \frac{2m}{r} \nonumber\\
&= 1 + \frac{v^2}{2c^2} - \frac{m}{r}.
\end{align}

Recordando que $m = GM/c^2$,
\begin{equation} \label{eq:energy-limit}
k \approx \frac{1}{M'c^2}\left( M'c^2 + \frac{1}{2} M' v^2 - \frac{GM M'}{r} \right),
\end{equation}
donde $M'$ es la masa de una partícula orbitando en la geodésica. Verificando así que en los límites mencionados $k$ se aproxima a la energía de la partícula por unidad de masa de su energía en reposo, pues en la ecuación \eqref{eq:energy-limit} tenemos la suma de la energía en reposo, la energía cinética y de la energía potencial gravitatoria (newtoniana).


Al igual que en el caso newtoniano (problema de Kepler), la órbita está contenida en un plano. Por la simetría del problema podemos elegir el plano ecuatorial: $\theta(0) = \pi/2$ y $\dot{\theta}(0) = 0$, entonces la ecuación \eqref{eq:geod-time-3} implica que $\ddot{\theta}(0) = 0$. Por tanto, para todo $\tau$,
\begin{equation} \label{eq:theta-sol}
\theta(\tau) = \frac{\pi}{2}.
\end{equation}

Nos queda por resolver la ecuación radial \eqref{eq:geod-time-1}, para ello usemos la identidad
\colorlet{shadecolor}{red!20}
\begin{shaded}
\begin{equation} 
   g_{\mu\nu} \frac{dx^{\mu}}{d\tau}\frac{dx^{\nu}}{d\tau} \equiv c^2.
\end{equation}
\end{shaded}

Expandiendo esta identidad, encontramos
\begin{align}
g_{\mu\nu} \frac{dx^{\mu}}{d\tau}\frac{dx^{\nu}}{d\tau} &= c^2 \\
g_{00} c^2 \dot{t}^2 + g_{11} \dot{r}^2 + g_{22} \dot{\theta}^2 + g_{33} \dot{\varphi}^2 &= c^2 \\
\left(1 - \frac{2m}{r} \right) c^2 \dot{t}^2 - \frac{1}{\left( 1- \frac{2m}{r} \right)} \dot{r}^2 - r^2 \CancelTo[\color{red}]{0}{\dot{\theta}^2} - r^2\sin^2\theta  \dot{\varphi}^2 &= c^2 \\
\left(1 - \frac{2m}{r} \right) c^2 \dot{t}^2 - \frac{\dot{r}^2 }{\left( 1- \frac{2m}{r} \right)}  - r^2 \dot{\varphi}^2 &= c^2. \label{eq:geodesic-time-5}
\end{align}
    
Usando \eqref{eq:Cantidades-Conservadas} y \eqref{eq:theta-sol}, podemos escribir \eqref{eq:geod-time-1} como
\begin{align}
\Ddot{r} + \frac{mc^2(r-2m)}{r^3} \dot{t}^2 - \frac{m}{r(r-2m)} \dot{r}^2 - (r-2m) \left[ \CancelTo[\color{red}]{0}{\dot{\theta}^2} + \sin^2\theta \dot{\varphi}^2 \right]  &= 0 \\
\Ddot{r} + \frac{mc^2(r-2m)}{r^3} \frac{k^2}{\left( 1 - \frac{2m}{r} \right)^2} - \frac{m}{r(r-2m)} \dot{r}^2 - (r-2m) \frac{h^2m^2c^2}{r^4}  &= 0 \\
\Ddot{r} + \frac{mc^2(r-2m)}{r^3} \frac{r^2k^2}{\left(r - 2m\right)^2} - \frac{m}{r(r-2m)} \dot{r}^2 - h^2m^2c^2 \frac{(r-2m)}{r^4}  &= 0 \\
\Ddot{r} + \frac{mc^2k^2}{r(r-2m)}  - \frac{m}{r(r-2m)} \dot{r}^2 - h^2m^2c^2 \frac{(r-2m)}{r^4}  &= 0
\end{align}

Por otro lado, podemos escribir \eqref{eq:geodesic-time-5} como
\begin{align}
\left(1 - \frac{2m}{r} \right) c^2 \dot{t}^2 - \frac{\dot{r}^2 }{\left( 1- \frac{2m}{r} \right)}  - r^2 \dot{\varphi}^2 &= c^2 \\
\left(1 - \frac{2m}{r} \right) \frac{c^2 k^2}{\left(1 - \frac{2m}{r} \right)^2} - \frac{\dot{r}^2}{\left(1 - \frac{2m}{r} \right)} - r^2 \frac{h^2m^2c^2}{r^4 \sin^4(\pi/2)} &= c^2 \\
\frac{c^2k^2}{\left(1 - \frac{2m}{r} \right)} - \frac{\dot{r}^2}{\left(1 - \frac{2m}{r} \right)}  -  \frac{h^2m^2c^2}{r^2} &= c^2 \\
\frac{\dot{r}^2}{\left(1 - \frac{2m}{r} \right)} &= \frac{k^2c^2}{\left(1 - \frac{2m}{r} \right)} - c^2 \left(1 + \frac{h^2m^2}{r^2} \right) \\
\dot{r}^2 &= k^2c^2 - c^2 \left(1 - \frac{2m}{r} \right)\left(1 + \frac{h^2m^2}{r^2} \right).
\end{align}

En resumen, la dinámica de la coordenada radial está dada por las siguientes ecuaciones:
\colorlet{shadecolor}{green!20}
\begin{shaded}
\begin{align}
\Ddot{r} &+ \frac{mc^2k^2}{r(r-2m)}  - \frac{m}{r(r-2m)} \dot{r}^2 - h^2m^2c^2 \frac{(r-2m)}{r^4}  = 0, \label{eq:geodesic-time-6} \\
\dot{r}^2 &= k^2c^2 - c^2 \left(1 - \frac{2m}{r} \right)\left(1 + \frac{h^2m^2}{r^2} \right).\label{eq:geodesic-time-7} 
\end{align}
\end{shaded}

Sin embargo, la ecuación \eqref{eq:geodesic-time-7} implica la condición \eqref{eq:geodesic-time-6}. En efecto, derivando con respecto a $\tau$ la ecuación \eqref{eq:geodesic-time-7}, tenemos que
\begin{align}
2 \dot{r} \ddot{r} &= - \frac{2m c^2}{r^2} \dot{r} \left(1 + \frac{h^2m^2}{r^2} \right) - c^2 \left(1 - \frac{2m}{r} \right)\left(- \frac{2h^2m^2}{r^3} \dot{r} \right) \\
\ddot{r} &= - \frac{mc^2}{r^2} \left(1 + \frac{h^2m^2}{r^2} \right) + h^2m^2c^2 \frac{(r-2m)}{r^4}. \label{eq:geodesic-time-8}
\end{align}

De la misma ecuación \eqref{eq:geodesic-time-7} se tiene que
\begin{equation}
- c^2 \left( 1 + \frac{h^2m^2}{r^2}\right) = \frac{\dot{r}^2 - k^2c^2}{\left( 1 - \frac{2m}{r}\right)}.
\end{equation}

Reemplazando en \eqref{eq:geodesic-time-8}:
\begin{align}
\ddot{r} &= - \frac{m}{r^2} \frac{\dot{r}^2 - k^2c^2}{\left(1 - \frac{2m}{r} \right)} + ^2m^2c^2 \frac{(r-2m)}{r^4} \\
\ddot{r} &= - \frac{mc^2k^2}{r(r-2m)} + \frac{m}{r(r-2m)} \dot{r}^2 + h^2m^2c^2 \frac{(r-2m)}{r^4}.
\end{align}

Por lo tanto, sólo es necesario resolver la ecuación \eqref{eq:geodesic-time-7}.

\section{Formulación Lagrangiana y Hamiltoniana}


Consideremos el lagrangiano efectivo (tarea 04 - Teoría General de la Relatividad 2023-01):
\begin{equation}
L = \frac{1}{2} g_{\mu\nu} \frac{dx^{\mu}}{d\tau}\frac{dx^{\nu}}{d\tau}.
\end{equation}

Evaluando las componentes de la métrica:
\begin{shaded}
\begin{equation}
L = \frac{1}{2} \left( 1 - \frac{2m}{r}\right)c^2 \dot{t}^2 - \frac{1}{2} \frac{\dot{r}^2}{\left( 1 - \frac{2m}{r}\right) }  - \frac{1}{2} r^2 \dot{\theta}^2 - \frac{1}{2} r^2\sin^2\theta \dot{\varphi}^2.
\end{equation}
\end{shaded}

Los momentos conjugados para las coordenadas generalizadas $x^{\mu} = (ct,r,\theta,\varphi)$ son 
\begin{align}
p_t &= \frac{\partial L}{\partial (c \dot{t})} = \left( 1 - \frac{2m}{r}\right)c \dot{t}, \label{eq:momento-1} \\
p_r &= \frac{\partial L}{\partial \dot{r}} = - \frac{\dot{r}}{\left( 1 - \frac{2m}{r}\right)}, \label{eq:momento-2}\\
p_{\theta} &= \frac{\partial L}{\partial \dot{\theta}} = - r^2\dot{\theta}, \label{eq:momento-3}\\
p_{\varphi} &= \frac{\partial L}{\partial \dot{\varphi}} = - r^2\sin^2\theta \dot{\varphi}. \label{eq:momento-4}
\end{align}

Como el lagrangiano no depende explícitamente de las coordenadas $ct$ y $r$, éstas son coordenadas cíclicas y por tanto los momentos conjugados $p_t$ y $p_{\varphi}$ son constantes en $\tau$. Además, comparando con \eqref{eq:Cantidades-Conservadas}, 
\begin{equation}
k = \frac{p_t}{c}, \quad hmc = - p_{\varphi}.
\end{equation}

Por otro lado, el hamiltoniano es
\begin{align}
H &= p_t c\dot{t} + p_r \dot{r} + p_{\theta} \dot{\theta} + p_{\varphi} \dot{\varphi} - L \nonumber \\
&= p_t c\dot{t} + p_r \dot{r} + p_{\theta} \dot{\theta} + p_{\varphi} \dot{\varphi} - \frac{1}{2} \left( 1 - \frac{2m}{r}\right)c^2 \dot{t}^2 + \frac{1}{2} \frac{\dot{r}^2}{\left( 1 - \frac{2m}{r}\right) }  + \frac{1}{2} r^2 \dot{\theta}^2 + \frac{1}{2} r^2\sin^2\theta \dot{\varphi}^2. \label{eq:pre-Hamiltonian}
\end{align}

Pero el hamiltoniano solo depende de las coordenadas generalizadas y sus respectivos momentos conjugados. Entonces, despejando las velocidades generalizadas en \eqref{eq:momento-1}-\eqref{eq:momento-4}, tenemos que
\begin{align}
 c \dot{t} &= \frac{p_t}{\left( 1 - \frac{2m}{r}\right)}, \label{eq:vel-1} \\
\dot{r} &=  - \left( 1 - \frac{2m}{r}\right) p_r, \label{eq:vel-2}\\
\dot{\theta} &= - \frac{p_{\theta}}{r^2}, \label{eq:vel-3} \\
\dot{\varphi} &= - \frac{p_{\varphi}}{r^2\sin^2\theta}. \label{eq:vel-4}
\end{align}

Reemplazando \eqref{eq:vel-1}-\eqref{eq:vel-4} en \eqref{eq:pre-Hamiltonian}:
\begin{align}
H &= \frac{p_t^2}{\left(1 -\frac{2m}{r} \right)} - \left(1 -\frac{2m}{r} \right) p_r^2 - \frac{p_{\theta}^2}{r^2} - \frac{p_{\varphi}^2}{r^2\sin^2\theta} \nonumber \\
&\quad - \frac{1}{2} \left( 1 - \frac{2m}{r}\right)\frac{p_t^2}{\left( 1 - \frac{2m}{r}\right)^2} + \frac{1}{2}\left( 1 - \frac{2m}{r}\right)^2 \frac{p_r^2}{\left( 1 - \frac{2m}{r}\right) }  + \frac{1}{2} r^2 \frac{p_{\theta}^2}{r^4} + \frac{1}{2} r^2\sin^2\theta \frac{p_{\varphi}^2}{r^4\sin^4\theta} \nonumber \\
&= \frac{p_t^2}{\left(1 -\frac{2m}{r} \right)} - \left(1 -\frac{2m}{r} \right) p_r^2 - \frac{p_{\theta}^2}{r^2} - \frac{p_{\varphi}^2}{r^2\sin^2\theta} - \frac{1}{2}\frac{p_t^2}{\left( 1 - \frac{2m}{r}\right)} + \frac{1}{2}\left( 1 - \frac{2m}{r}\right) p_r^2  + \frac{1}{2}  \frac{p_{\theta}^2}{r^2} + \frac{1}{2} \frac{p_{\varphi}^2}{r^2\sin^2\theta} \nonumber \\
&= \frac{1}{2}\frac{p_t^2}{\left(1 -\frac{2m}{r} \right)} - \frac{1}{2} \left(1 -\frac{2m}{r} \right) p_r^2 - \frac{1}{2}\frac{p_{\theta}^2}{r^2} - \frac{1}{2} \frac{p_{\varphi}^2}{r^2\sin^2\theta}.
\end{align}

Por lo tanto, el hamiltoniano está dado por
\begin{shaded}
\begin{equation}
H = \frac{1}{2}\frac{p_t^2}{\left(1 -\frac{2m}{r} \right)} - \frac{1}{2} \left(1 -\frac{2m}{r} \right) p_r^2 - \frac{1}{2}\frac{p_{\theta}^2}{r^2} - \frac{1}{2} \frac{p_{\varphi}^2}{r^2\sin^2\theta}.
\end{equation}
\end{shaded}

Las ecuaciones de Hamilton son:
\begin{align}
c \dot{t} &= \frac{\partial H}{\partial p_t} = \frac{p_t}{\left( 1 - \frac{2m}{r}\right)}, \label{eq:Hamilton-1} \\
\dot{r} &= \frac{\partial H}{\partial p_r} = -  \left(1 -\frac{2m}{r} \right) p_r \label{eq:Hamilton-2}, \\
\dot{\theta} &= \frac{\partial H}{\partial p_{\theta}} = - \frac{p_{\theta}}{r^2} \label{eq:Hamilton-3}, \\
\dot{\varphi} &= \frac{\partial H}{\partial p_{\varphi}} = - \frac{p_{\varphi}}{r^2\sin^2\theta}, \label{eq:Hamilton-4}\\
\dot{p}_t &= - \frac{\partial H}{\partial (ct)} = 0, \label{eq:Hamilton-5}\\
\dot{p}_r &= - \frac{\partial H}{\partial r} = \frac{m}{(r - 2m)^2} p_t^2 + \frac{m}{r^2} p_r^2 - \frac{p_{\theta}^2}{r^3} - \frac{p_{\varphi}^2}{r^3\sin^2\theta}, \label{eq:Hamilton-6} \\
\dot{p}_{\theta} &= - \frac{\partial H}{\partial \theta} =  \frac{\cos\theta }{r^2 \sin^3\theta} p_{\varphi}^2,  \label{eq:Hamilton-7} \\
\dot{p}_{\varphi} &= - \frac{\partial H}{\partial \varphi} = 0. \label{eq:Hamilton-8}
\end{align}


Las ecuaciones \eqref{eq:Hamilton-5} y \eqref{eq:Hamilton-8} nos dicen lo que ya sabíamos, que hay cantidades conservadas asociadas a la energía y al momento angular. Estas cantidades se relacionan con la primera derivada de las coordenadas $t$ y $\varphi$ dadas por las ecuaciones \eqref{eq:Hamilton-1} y \eqref{eq:Hamilton-4}. Mostremos que las ecuaciones \eqref{eq:Hamilton-2} y \eqref{eq:Hamilton-6} implican \eqref{eq:geod-time-2} y que las ecuaciones \eqref{eq:Hamilton-3} y \eqref{eq:Hamilton-7} implican \eqref{eq:geod-time-3}. 

Derivando la ecuación \eqref{eq:Hamilton-2} con respecto a $\tau$:
\begin{equation} \label{eq:Hamilton-9}
\ddot{r} = - \frac{2m}{r^2} \dot{r} p_r - \left( 1 - \frac{2m}{r}\right) \dot{p}_r.
\end{equation} 

Reemplazando la ecuación \eqref{eq:Hamilton-6} en \eqref{eq:Hamilton-9}:
\begin{equation}
\ddot{r} = - \frac{2m}{r^2} \dot{r} p_r - \left( 1 - \frac{2m}{r}\right) \left( \frac{m}{(r - 2m)^2} p_t^2 + \frac{m}{r^2} p_r^2 - \frac{p_{\theta}^2}{r^3} - \frac{p_{\varphi}^2}{r^3\sin^2\theta}  \right).
\end{equation} 

Sustituyendo las expresiones para los momentos conjugados \eqref{eq:momento-1}-\eqref{eq:momento-4}:
\begin{align}
\ddot{r} &= \frac{2m}{r^2} \frac{\dot{r}^2}{\left( 1 - \frac{2m}{r}\right)} - \left( 1 - \frac{2m}{r}\right) \left( \frac{m}{(r - 2m)^2}  \left( 1 - \frac{2m}{r}\right)^2 c^2 \dot{t}^2 + \frac{m}{r^2} \frac{\dot{r}^2}{\left( 1 - \frac{2m}{r}\right)^2} - \frac{r^4\dot{\theta}^2}{r^3} - \frac{r^4 \sin^4\theta \dot{\varphi}^2}{r^3\sin^2\theta}  \right)  \nonumber \\
&= \frac{2m}{r^2} \frac{\dot{r}^2}{\left( 1 - \frac{2m}{r}\right)} - \frac{(r-2m)}{r} \frac{m}{(r-2m)^2} \frac{(r-2m)^2}{r^2} c^2 \dot{t}^2 - \frac{m}{r^2} \frac{\dot{r}^2}{\left( 1 - \frac{2m}{r}\right)} + \frac{(r-2m)}{r} r \dot{\theta}^2 + \frac{(r-2m)}{r} r \sin^2\theta \dot{\varphi}^2 \nonumber \\
&= \frac{m}{r(r-2m)} \dot{r}^2 - \frac{mc^2(r-2m)}{r^3} \dot{t}^2 + (r-2m)\left[ \dot{\theta}^2 + \sin^2\theta \dot{\varphi}^2\right]
\end{align}

Ahora, derivando la ecuación \eqref{eq:Hamilton-3} con respecto a $\tau$:
\begin{equation} \label{eq:Hamilton-10}
\ddot{\theta} = \frac{2\dot{r} p_{\theta}}{r^3} + \frac{\dot{p}_{\theta}}{r^2}.
\end{equation} 

Reemplazando la ecuación \eqref{eq:Hamilton-7} en \eqref{eq:Hamilton-10}:
\begin{equation}
\ddot{\theta} = \frac{2\dot{r} p_{\theta}}{r^3} +  \frac{\cos\theta }{r^4 \sin^3\theta}p_{\varphi}^2 .
\end{equation}

Sustituyendo las expresiones para los momentos conjugados \eqref{eq:momento-3} y \eqref{eq:momento-4}:
\begin{align}
\ddot{\theta} &= - \frac{2 r^2\dot{r} \dot{\theta}}{r^3} +  \frac{r^4\sin^4\theta \cos\theta \dot{\varphi}^2}{r^4 \sin^3\theta} \nonumber  \\
&= - \frac{2}{r} \dot{r}\dot{\theta} +  \sin\theta\cos\theta \dot{\varphi}^2.
\end{align}


\end{document}

